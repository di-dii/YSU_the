% !Mode:: "TeX:UTF-8"
\chapter{表格}
\label{chap:table}

\section{普通三线表}
科技文献中常用的三线表:
\begin{table}[htbp]
 \centering\zihao{5}
 \caption{燕山大学硕士学位论文参考文献规则}\label{tab:ysubof}
 \begin{tabular}{llr}
 \toprule
    论文版本    & 参考文献标准    & 实施年份(年)  \\
 \midrule
    旧版        & BF7714-87       & 1987            \\
    新版        & GBT7714-2005    & 2005            \\
 \bottomrule
 \end{tabular}
\end{table}

实现代码如下:
\begin{verbatim}
\begin{table}[htbp]
 \centering\zihao{5}
 \caption{燕山大学硕士学位论文参考文献规则}\label{tab:ysubof}
 \begin{tabular}{llr}
 \toprule
    论文版本    & 参考文献标准    & 实施年份(年)  \\
 \midrule
    旧版        & BF7714-87       & 1987            \\
    新版        & GBT7714-2005    & 2005            \\
 \bottomrule
 \end{tabular}
\end{table}
\end{verbatim}

\section{有合并列的三线表}
合并列通常见于表格的第一行,在适当的位置使用\verb|\multicolumn| 命令即可。
\begin{table}[htbp]
\centering\zihao{5}
\caption{带有合并列的三线{\zihao{5}表}}\label{tab:test}
\begin{tabular}{llr} \toprule
\multicolumn{2}{c}{Item} \\ \cmidrule(r){1-2}
Animal & Description & Price (\$)\\ \midrule
Gnat & per gram & 13.65 \\
& each & 0.01 \\
Gnu & stuffed & 92.50 \\
Emu & stuffed & 33.33 \\
Armadillo & frozen & 8.99 \\ \bottomrule
\end{tabular}
\end{table}


该表格是采用如下代码实现的:
\begin{verbatim}
\begin{table}[htbp]
 \centering
 \caption{三线表}\label{tab:test}
 \begin{tabular}{llr}
 \toprule
    \multicolumn{2}{c}{Item}            \\
 \cmidrule(r){1-2}
    Animal  & Description   & Price (\$)\\
 \midrule
    Gnat    & per gram      & 13.65     \\
            & each          & 0.01      \\
    Gnu     & stuffed       & 92.50     \\
    Emu     & stuffed       & 33.33     \\
    Armadillo & frozen      & 8.99      \\
 \bottomrule
 \end{tabular}
\end{table}
\end{verbatim}

\section{表格的引用}
表格的引用同样是使用\verb|\ref{}| 命令实现的。例如“表\verb|\ref{tab:ysubof}|” 输出的结果为:表\ref{tab:ysubof}
